\# Bench Parser \#

Parser for the I\+S\+C\+A\+S85/89/99 bench file format. It parsers the bench files aiming at generating an abstract syntax graph (A\+SG). The nodes of the A\+SG are topologically sorted in order to make the calls for the B\+DD package.

\subsection*{Installation}

\subsubsection*{Step 1\+: Clone the directory}

Run the git clone command replacing $\ast$$<$username$>$$\ast$ by you username. 
\begin{DoxyCode}
1 git clone https://<username>@bitbucket.org/cpnogueira/bench\_parser.git
\end{DoxyCode}


\subsubsection*{Step 2\+: Compiling}

To compile it, after cloning it from bitbucket, you should be in the root folder of the project and run the command\+: 
\begin{DoxyCode}
1 make
\end{DoxyCode}


To clean files produced during compilation (object files, executables, libraries, ...) type\+: 
\begin{DoxyCode}
1 make clean
\end{DoxyCode}


\subsection*{Running the executable}

The different designs for the bechmarks are located in the directory \char`\"{}benchmarks\char`\"{} For example, to run the code for the circuit s420.\+bench of the iscas89 benchmark type\+: 
\begin{DoxyCode}
1 ./bench\_parser benchmarks/iscas89/s420.bench
\end{DoxyCode}


This will produce the subfolder results\+\_\+s420 containing the results.

\subsection*{A\+PI Reference}

To generate or update the documentation, you should navigate to the doc/ directory and then\+: 
\begin{DoxyCode}
1 ../bench\_parser/doc/$ doxygen doxyConfigEx
\end{DoxyCode}
 The documentation will be available on $\ast$../bench\+\_\+parser/doc/doxygen/$\ast$ and you can access it from the same location opening the file {\itshape index.\+html}. 